\documentclass[12pt]{article}
\usepackage[utf8]{inputenc}

\usepackage{amsmath}
\usepackage{amssymb}
\usepackage{amsthm}
\usepackage{float}
\usepackage{graphicx}
\usepackage{geometry}
\usepackage{nicefrac}
\usepackage{color}
%% For fixing Unicode characters in pdf_tex file from Inkscape
% \DeclareUnicodeCharacter{00A0}{ }
% \usepackage[Smaller]{cancel}
% Unused packages
    % \usepackage{comment}
    % \usepackage{longtable}
    % \usepackage{booktabs}
    % \usepackage{gensymb}
    % \usepackage{subcaption}
    % \usepackage[hyphens]{url}
    % \usepackage{hyperref}
    % \hypersetup{breaklinks=true}
    % \usepackage{trimclip}
    % \usepackage{units}
    % \usepackage{listings}

\newtheorem{proposition}{Proposition}
\newtheorem{theorem}{Theorem}
% \newtheorem{proof}{Proof}

\geometry{a4paper,total={170mm,257mm},left=20mm,top=20mm}

\title{\vspace{-2cm} Proof : Rolling Around\vspace{-1.5ex}}
\author{Sai  Theja K.}
\date{}

\begin{document}

\maketitle
\vspace*{-3em}
\begin{proposition}
    Locus of a point on a smaller circle (radius = $R$) rolling (without slipping) on the inner surface of a larger circle (radius = $2R$) is a straight line through the center of the larger circle.
\end{proposition}

\begin{figure}[h]
    \centering
    \def\svgwidth{0.75\linewidth}
    \graphicspath{{../inkscape/tex_pdf/}} % can be skipped if figure is in the same folder
    \input{../inkscape/tex_pdf/drawing.pdf_tex} % skip <path to file>/ if figure is in the same folder
    \caption{Schematic of motion}
    \label{fig:schematic}
\end{figure}

\begin{proof}
    Let the point, center of smaller circle and the center of the larger circle (origin) (say $P$, $C$ and $O$ respectively) be represented in cylindrical co-ordinates by $(r_P, \theta_P)$, $(r_C, \theta_C)$ and $(r_O, \theta_O)$. The schematic is shown in Figure \ref{fig:schematic}. By definition, $(r_O, \theta_O) = (0, 0)$
    \section{Locus of $P$}
    Let's consider a point $P$ on the smaller circle whose radius makes an angle $\alpha$ with the x-axis. $P$ is represented as $(r_P, \theta_P)$ and let us derive the expression for $r_P$ during this rolling motion.\\
    From $\triangle OCP$, $\overline{OP}(r_P)$ can be calculated using cosine law.
    \begin{equation}
        {\overline{OP}}^2 = {\overline{OC}}^2 + {\overline{CP}}^2 - 2 \; \overline{OC} \; \overline{CP} \; \cos(\angle OCP) \label{eqn:cosine_law}
    \end{equation}
    $\angle OCP$ can be calculated as the other two angles in $\triangle OCP$ are equal (isosceles triangle) to $\left( \theta_C-\theta_P \right)$.
    \begin{gather*}
        \overline{OC} = R = \overline{PC} \implies \angle CPO = \angle POC = \theta_C-\theta_P \\
        \therefore \angle OCP = \pi - \angle CPO - \angle POC = \pi - 2 \left(\theta_C-\theta_P\right)
        % r^2_P = 
    \end{gather*}
    $r_P$ can be calculated from Equation \ref{eqn:cosine_law} as follows :
    \begin{align}
        {r_P}^2 &= R^2 + R^2 - 2 \times R \times R \times \cos(\pi - 2 \left(\theta_C-\theta_P\right)) \nonumber \\
                &= 2 R^2 \left\{ 1 + \cos\left(2 \left(\theta_C-\theta_P\right)\right) \right\} = 2 R^2 \left\{ 2 \cos^2\left(\theta_C-\theta_P\right) \right\} \nonumber \\
        \implies &\boxed{r_P = 2R\cos\left(\theta_C-\theta_P\right)} \label{eqn:locus_r}
    \end{align}
    \subsection{Relation between $\omega_{CO}$ and $\omega_{PC}$}
    Consider the point $T$ on the smaller circle in contact with the larger circle. The no-slip condition dictates that there will be no relative motion between the surfaces at the point of contact $T$.
    \begin{equation}
        \vec{V}_{TO} = 0 = \vec{V}_{TC} + \vec{V}_{CO} \tag{No-Slip Condition} \label{eqn:no-slip}
    \end{equation}
    This condition gives us a relation between the angular velocities $\vec{\omega}_{CO} \; (\dot{\theta}_C)$ and $\vec{\omega}_{TC} \; (\dot{\alpha})$ :
    \begin{align}
        \vec{\omega}_{CO} &= \vec{r}_{CO} \times  \vec{V}_{CO} \nonumber \\
        \vec{\omega}_{TC} &= \vec{r}_{TC} \times  \vec{V}_{TC} = \vec{r}_{TC} \times -\vec{V}_{CO} = -\vec{\omega}_{CO} \nonumber \\
        \vec{\omega}_{TC} &= \boxed{\dfrac{d \alpha}{d t} = -\dfrac{d \theta_C}{d t}} = -\vec{\omega}_{CO} \label{eqn:alpha_dot}
    \end{align}
    For deriving $\alpha$, consider the following in $\triangle CNP$,
    \begin{itemize}
        \item $\angle CPN = \angle CPO = \theta_C - \theta_P$
        \item $\angle CNP$ and $\theta_P$ are supplementary angles $\implies \angle CNP = \pi - \theta_P$
        \item Thus, we can find the third angle $\angle PCN \, (\alpha)$
    \end{itemize}
    \begin{equation}
        \angle PCN = \boxed{\alpha = 2 \, \theta_P - \theta_C} = \pi - \angle CPN - \angle CNP \label{eqn:alpha}
    \end{equation}
    From Equations \ref{eqn:alpha_dot} and \ref{eqn:alpha} :
    \begin{gather}
        2\dfrac{d \theta_P}{d t} - \dfrac{d \theta_C}{d t} = \dfrac{d \alpha}{d t} = -\dfrac{d \theta_C}{d t} \implies \boxed{\dfrac{d \theta_P}{d t} = 0} \label{eqn:locus_theta}
    \end{gather}
    \section{Conclusion}
    For the trajectory of $P$ i.e., time evolution of position of $P$, consider a general case where initially $\theta_C(t=0) = \theta^0_C$, $\theta_P(t=0) = \theta^0_P$ and $\dot{\theta}_C = \omega \text{ (constant)}$
    \begin{gather}
        \boxed{\theta_P = \theta^0_P} \label{eqn:final_locus_theta} \\
        \theta_C = \theta^0_C + \omega \, t \nonumber \\
        r_P = 2 R \cos \left( \theta_C-\theta_P \right) \nonumber \\
        \implies \boxed{r_P = 2 R \cos \left( \theta^0_C + \omega \, t-\theta^0_P \right)} \label{eqn:final_locus_r}
    \end{gather}
    From Equation \ref{eqn:final_locus_theta}, we can see that the angle $P$ makes with the $x$-axis ($\theta_P$) remains constant and from Equation \ref{eqn:final_locus_r} that $r_P$ goes to zero periodically. Therefore, $P$ passes through origin $O$ while maintaining a constant slope $\left(\tan(\theta_P)\right)$ and thus it \emph{slides on a straight line through the origin} during the rolling motion as proposed originally.
\end{proof}

% \begin{itemize}
    % \item alpha theta c theta p relation then theta c constant proof
    % \item Finalise the locus
    % \item Relocate the text anchors properly - Re-export PDF+Latex way
    % \item Take care of (2R, R) vs (R, R/2)
    % \item Add R on CT
    % \item 1. omega relation, 2. locus of general point
% \end{itemize}

\end{document}
